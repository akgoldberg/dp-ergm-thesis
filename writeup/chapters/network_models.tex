
An increasingly popular approach in quantitative analysis of networks is to fit statistical models to realized network data. Many of these models have generative interpretations, allowing researchers to understand the relative importance of multiple endogenous processes to the resulting structure of the network. The advantage of such an approach is best illustrated in contrast to computing statistics -- like degree distributions or clustering coefficients -- to describe the network structure, without an explicit model of the network. While such metrics are useful in summarizing the structural properties of a given network, they cannot tease out the underlying processes that may give rise to such structures. 

For example, one of the distinguishing characteristics of many real-world social networks is the tendency to have more triangles (sets of three connected nodes) than would be expected by drawing random edges of a graph \cite{GKM09}. There are a number of different processes in the formation of a friend network that could give rise to this outcome. One potential explanation is the notion of ``triangle closure,'' or the tendency for people to become friends with friends-of-friends, since they are easier to meet. Another subtly different explanation is that triangles arise out of ``assortative matching,'' the propensity for people with the same attributes to become friends with one another, leading to clustering in the network. Finally, a high number of triangles in a social network could arise for reasons of ``sociality,''  the presence of only a few highly social individuals in the network, who are mutual friends to many people.

In order to consider what global or local processes best explain particular structures of a network, a statistical model of network data posits a probability distribution over the space of possible networks. The goal of inference  is to tune parameters of the distribution, such that the realized network is likely to be observed under the probability distribution. 

A simple example of such a model is the Erd\"{o}s-R\'{e}nyi Random Graph Model, known as the $G(n,p)$ model, which proposes that edges are drawn independently with probability $p$ between any two nodes of a network with $n$ nodes. While this model has been studied in great depth by graph theorists, it does not capture many important features of real world networks, like the tendency for clustering or the power-law distribution of degrees. In order to model such structures in networks, a more general class of random graph models are Exponential Random Graph Models.

\section{Exponential Random Graph Models (ERGMs)}\label{sec:ergms}

Formally, a graph $G = (V,E)$ is defined by a set of nodes (or vertices) $V$, with $|V| = n$ and edges $E$, representing the presence or absence of relationships between nodes. We will use the ``adjacency matrix'' representation of a graph, which we denote $x$, where $x_{ij} = 1$ if an edge exists between nodes $i$ and $j$ and $x_{ij} = 0$ otherwise.  The models we consider are defined over undirected graphs, where all the edges are bidirectional, and the adjacency matrix is therefore symmetric. We refer to the number of edges adjacent to node $i$ as the \emph{degree} of node $i$ so $d_i = \sum_{j=1}^{n} x_{ij}$. Then, the \emph{degree distribution} is $D = (D_0,...,D_{n-1})$ where $D_k = |\{i \in V : d_i = k \}|$.

\begin{definition}[Exponential Random Graph \cite{WS96}]
\label{def:ergm_defn}
A probability distribution over graphs of $n$ vertices belongs to the family of \emph{exponential random graph models} (henceforth referred to as ERGMs) if it takes the form:
\begin{align*}
%\label{eq:ergm}
\Pr(x | \theta) = \exp\left\{\theta^T u(x) - \psi(\theta)  \right\}
\end{align*}
where $\theta$ is a vector of parameters of the model, $u(x)$ is a vector of sufficient statistics computed on graph $x$, and $\psi(\theta)$ is a normalization constant needed to ensure a valid probability distribution so:
\begin{align*}
%\label{eq:ergm_norm_constant}
\psi(\theta) = \log \sum_{x'} \exp\left\{\theta^T u(x') \right\}
\end{align*}
\end{definition}

%  One advantage to this model is that it belongs to the  \emph{exponential family} of probability distributions, for which inference techniques are well studied in the statistics and machine learning literature. In general, the normalization constant $\psi(\theta)$ may be intractable to compute exactly since it requires summing over the space of all possible graphs on $n$ vertices. Therefore, in practice, approximate inference methods, in particular sampling-based MCMC approaches, are used for parameter estimation of these models on realized data. 
 
ERGMs  describe a broad class of random graphs, with varying conditional dependence relationships between edges.  For instance, the $G(n,p)$ graph can be viewed as an ERGM:
 
 \begin{example}[$G(n,p)$ graphs]
 	\label{ex:ER_model}
 We can represent the Erd\"{o}s-R\'{e}nyi Random Graph ($G(n,p)$) model as an ERGM, by taking 
 $$u(x) = |E| \text{, } \quad
 \theta = \log \frac{p}{1-p}$$ %\text{, } \quad 
  $$\psi(\theta)  = -\binom{n}{2} \log(1-p) = -\binom{n}{2} \log \frac{e^{-\theta}}{1+ e^{-\theta}}
  $$
  Then, 
  \begin{align*}
    \Pr(x|\theta) & = \exp\left\{ |E| \log \frac{p}{1-p} + \binom{n}{2} \log(1-p)  \right\}  \\
    & = p^{|E|} (1-p)^{\binom{n}{2} - |E|}\\
    & = \prod_{i < j} p^{x_{ij}} (1-p)^{1 - x_{ij}}
  \end{align*}
  so each possible edge is included independently with probability $p$ as specified by the Erd\"{o}s-R\'{e}nyi Model.
 \end{example}

In order to model more complex structures in a network,  researchers have proposed higher order sufficient statistics of ERGMs that imply more general conditional independence assumptions than the Erd\"{o}s-R\'{e}nyi Model. For instance,  ``Markov'' graphs, allow the probabilities of any two possible edges in a graph to be conditionally dependent if the edges share a common endpoint. This dependency allows for node level effects on edge formation. In fact, Markov dependencies are captured by ERGMs of the following form:

\begin{example}[Markov graphs \cite{FS86}]
\label{ex:markov_graphs}
Any undirected \emph{Markov graph} has probability distribution:
\begin{equation*}
\label{eqn:markov_graph}
\Pr(x | \theta, \tau) = \exp\left\{ \sum_{k = 1}^{n-1} \theta_k S_k(x) + \tau T(x) - \psi(\theta, \tau)   \right\} 
\end{equation*}
where the sufficient statistics are
\begin{align*}
&\text{number of edges:} & S_1(x)  & = \sum_{1 \leq i < j \leq n} x_{ij} = |E|\\
&\text{number of $k$-stars } (k \geq 2): & S_k(x)  & = \sum_{i=1}^{n-1} \binom{i}{k} D_i(x)\\
&\text{number of triangles: } & T(x) & = \sum_{1 \leq h < i < j < \leq n} x_{hi} x_{ij} x_{hj}
\end{align*}
and the parameters are $\{\theta_k\}_{k=1}^{n}$ and $\tau$.\footnote{Note that setting $\theta_2 = ...= \theta_k = \tau = 0$ in the Markov model, we recover the $G(n,p)$ model, which is an instance of a Markov graph since any two edges are conditionally independent in the $G(n,p)$ model.}
\end{example}

%First, graphs instances of the general Markov graph model are susceptible to \emph{model degeneracy}, where we refer to a probability distribution as degenerate if its mass is concentrated on a small subset of the space of possible graphs. For instance, consider the Markov graph  with sufficient statistics $S_1$ and $T$ (so $\theta_k = 0$ for $k \geq 2$). If we take $\theta_1 > 0$ and let $\tau$ be fairly large and positive, then this model puts almost all of its mass on the complete graph or nearly complete graphs, since the term in the exponent is extremely large for such graphs, as there are $\binom{n}{3}$ possible triangles. In fact, for $\tau > 0$, this model asymptotically (as $n \to \infty$) results in only three possible distributions: (1) all probability mass on the complete graph, (2) the $G(n,p)$ graph model or (3) a mixture distribution with some probability of the complete graph and some of $G(n,p)$ graphs \cite{Jon99}. Thus, because we do not expect most interesting real world social or biological networks to be complete or $G(n,p)$ graphs, this model in its current form is not conducive to modeling real world networks.

%Second, Markov graph models often suffer from \emph{inferential degeneracy}, or the existence of many parameters that could maximize the likelihood $\Pr(x | \theta, \tau)$. As a simple example of such an issue, consider the case where the maximum degree of the graph is bounded by some $k < n-1$ so that there are no $(k+1)$-stars, but the model contains $S_{k+1}(x)$ as sufficient statistics and a corresponding parameter $\theta_{k+1}$. Then the parameter $\theta_{k+1}$ could take on any value without changing the likelihood, so common techniques for maximum likelihood estimation may fail to converge. \cite{Han03}. 

%Lastly, the high sensitivity of the likelihood of general Markov graph models to addition or removal of edges makes common inference techniques challenging. As explained earlier, due to the intractability of computing the normalizing constant $\psi(\theta)$, sampling based inference methods are generally employed to perform inference over ERGMs. Roughly speaking, such methods proceed by sampling edges of a network in turn, holding the other edges constant, as repeating this procedure defines a Markov chain that converges asymptotically to the true distribution. At each sampling step, there is some probability of adding an edge to the graph which should be drawn according to the desired distribution. Now, because edges exhibit conditional dependencies, the addition of one edge may increase the probability of another edge being added to the graph. The difficulty with the specification of ERGMs given in $\Cref{ex:markov_graphs}$ is that the likelihood can be highly sensitive to the addition or removal of an edge. For instance, including high-order $k$-stars all with positive $\theta$, then for every additional edge added to a high degree node, the change to the likelihood grows exponentially since a $d$-degree node has $\binom{d}{k}$, $k$-stars. Thus, applying sampling based procedures to these Markov graph models tends to lead to ``avalanche'' effects -- as we add edges the conditional probability of other edges explosively increases, leading to convergence to the complete graph \cite{Sni+06}. In fact, this is related to the issue of model degeneracy, since most of the probability mass of the distribution is on the complete graph. Thus, alternative models aim to use sufficient statistics that have smaller impact on the sufficient statistics and thus the likelihood of the model. This notion of robustness of the sufficient statistics to addition or removal of edges is closely related to differential privacy, suggesting that the subsequent model may be amenable to use under differential privacy constraints.

\section{Sufficient Statistics of ERGMs}\label{sec:alt_stats}

In practice, due to its simplicity, the $G(n,p)$ model is used only as a starting point in inference over real-world data, while the full Markov graph model is infrequently used as it suffers from poor statistical properties. In particular, the Markov graph model is degenerate for many parameter configurations, representing only distributions that put all of their probability mass on either nearly-complete graphs (graphs with all edges present) or on $G(n,p)$ graphs \cite{Jon99}. In response to these problems of degeneracy with Markov graphs, more robust ``alternating'' sufficient statistics are generally used in ERGMs to capture structural properties of networks.  We will first provide definitions of these statistics and then expand on the mathematical motivation behind them.

\subsection{Alternating Sufficient Statistics}

\begin{figure}[!ht]
	\label{fig:graphdiagram}
	\centering
	%%%%%%%%%%%%%%%%%%%%%%%%%%%%%%%%%%%%%%%%%%%%%%%%%%%%%%%%%%%%
%%															k-stars 											           %%
%%%%%%%%%%%%%%%%%%%%%%%%%%%%%%%%%%%%%%%%%%%%%%%%%%%%%%%%%%%%
$k$-stars
\vspace{0.5cm}

\begin{tikzpicture}[roundnode/.style={circle, fill=black!60, very thick,minimum size=2mm, inner sep=1.2pt}]

%Nodes
\node[roundnode]      (u1)                     {};
\node[roundnode]      (u2)       [below=1cm of u1] {};

%Lines
\path  (u1) edge (u2);
\end{tikzpicture}
\qquad
\begin{tikzpicture}[roundnode/.style={circle, fill=black!60, very thick,minimum size=2mm, inner sep=1.2pt}]

%Nodes
\node[roundnode]      (u1)                     {};
\node[roundnode]      (u2)       [below right=1cm and 0.5cm of u1] {};
\node[roundnode]      (u3)       [below left=1cm and 0.5cm of u1] {};

%Lines
\path  (u1) edge (u2);
\path  (u1) edge (u3);
\end{tikzpicture}
\qquad
\begin{tikzpicture}[roundnode/.style={circle, fill=black!60, very thick,minimum size=2mm, inner sep=1.2pt}]

%Nodes
\node[roundnode]      (u1)                     {};
\node[roundnode]      (u2)       [below=0.5cm of u1] {};
\node[roundnode]      (u3)       [below right=0.5cm and 0.5cm of u2] {};
\node[roundnode]      (u4)       [below left=0.5cm and 0.5cm of u2] {};

%Lines
\path  (u2) edge (u1)
edge (u3)
edge (u4);
\end{tikzpicture}
\qquad
\begin{tikzpicture}[roundnode/.style={circle, fill=black!60, very thick,minimum size=2mm, inner sep=1.2pt}]

%Nodes
\node[roundnode]      (u1)                     {};
\node[roundnode]      (u2)       [below right=0.5cm and 0.5cm of u1] {};
\node[roundnode]      (u3)       [above right=0.5cm and 0.5cm of u2] {};
\node[roundnode]      (u4)       [below left=0.5cm and 0.5cm of u2] {};
\node[roundnode]      (u5)       [below right=0.5cm and 0.5cm of u2] {};

%Lines
\path  (u2) edge (u1)
				  edge (u3)
				  edge (u4)
				  edge (u5);
\end{tikzpicture}
\qquad
\begin{tikzpicture}[roundnode/.style={circle, fill=black!60, very thick,minimum size=2mm, inner sep=1.2pt}]

%Nodes
\node[roundnode]      (u1)                     {};
\node[roundnode]      (u2)       [below left=1.0cm and 0.8cm of u1] {};
\node[roundnode]      (u3)       [below left=1.0cm and 0.4cm of u1] {};
\node[roundnode]      (u4)       [below right=1.0cm and 0.4cm of u1] {};
\node[roundnode]      (u5)       [below right=1.0cm and 0.8cm of u1] {};
\node[]      (ellipses)       [below=0.9 cm of u1] {\ldots};

%Lines
\path  (u1) edge (u2)
edge (u3)
edge (u4)
edge (u5);
\end{tikzpicture}
   
\vspace{0.5cm}

%%%%%%%%%%%%%%%%%%%%%%%%%%%%%%%%%%%%%%%%%%%%%%%%%%%%%%%%%%%%
%%															k-triangles 											     %%
%%%%%%%%%%%%%%%%%%%%%%%%%%%%%%%%%%%%%%%%%%%%%%%%%%%%%%%%%%%%
$k$-triangles
\vspace{0.5cm}

\begin{tikzpicture}[roundnode/.style={circle, fill=black!60, very thick,minimum size=2mm, inner sep=1.2pt}]

%Nodes
\node[roundnode]      (u1)                     {};
\node[roundnode]      (u2)       [below=1cm of u1] {};
\node[roundnode]      (u3)       [below right=0.5cm and 0.8cm of u1] {};

%Lines
\path  (u1) edge (u2);

\path (u1) edge (u3);

\path (u2) edge (u3);
\end{tikzpicture}
\qquad
\begin{tikzpicture}[roundnode/.style={circle, fill=black!60, very thick,minimum size=2mm, inner sep=1.2pt}]
 
 %Nodes
 \node[roundnode]      (u1)                     {};
 \node[roundnode]      (u2)       [below=1cm of u1] {};
 \node[roundnode]      (u3)       [below right=0.5cm and 0.8cm of u1] {};
 \node[roundnode]      (u4)       [right=0.8cm of u3] {};
 
 %Lines
 \path  (u1) edge (u2);
 
 \path (u1) edge (u3)
 edge (u4);
 
 \path (u2) edge (u3)
 edge (u4);
 \end{tikzpicture}
 \qquad
  \begin{tikzpicture}[roundnode/.style={circle, fill=black!60, very thick,minimum size=2mm, inner sep=1.2pt}]
  
%Nodes
\node[roundnode]      (u1)                     {};
\node[roundnode]      (u2)       [below=1cm of u1] {};
\node[roundnode]      (u3)       [below right=0.5cm and 0.8cm of u1] {};
\node[roundnode]      (u4)       [right=0.8cm of u3] {};
\node[]      (ellipses)       [right=0.15cm of u4] {\ldots};
\node[roundnode]      (u5)       [right= 1.4cm of u4] {};

%Lines
\path  (u1) edge (u2);

\path (u1) edge (u3)
				 edge (u4)
				 edge (u5);

\path (u2) edge (u3)
				 edge (u4)
				 edge (u5);
\end{tikzpicture}

\vspace{0.5cm}

%%%%%%%%%%%%%%%%%%%%%%%%%%%%%%%%%%%%%%%%%%%%%%%%%%%%%%%%%%%%
%%															k-two paths 											    %%
%%%%%%%%%%%%%%%%%%%%%%%%%%%%%%%%%%%%%%%%%%%%%%%%%%%%%%%%%%%%
$k$-two-paths
\vspace{0.5cm}

\begin{tikzpicture}[roundnode/.style={circle, fill=black!60, very thick,minimum size=2mm, inner sep=1.2pt}]

%Nodes
\node[roundnode]      (u1)                     {};
\node[roundnode]      (u2)       [below=1cm of u1] {};
\node[roundnode]      (u3)       [below right=0.5cm and 0.8cm of u1] {};

%Lines
\path[dotted] (u1) edge (u2);

\path (u1) edge (u3);

\path (u2) edge (u3);
\end{tikzpicture}
\qquad
\begin{tikzpicture}[roundnode/.style={circle, fill=black!60, very thick,minimum size=2mm, inner sep=1.2pt}]

%Nodes
\node[roundnode]      (u1)                     {};
\node[roundnode]      (u2)       [below=1cm of u1] {};
\node[roundnode]      (u3)       [below right=0.5cm and 0.8cm of u1] {};
\node[roundnode]      (u4)       [right=0.8cm of u3] {};

%Lines
\path[dotted] (u1) edge (u2);

\path (u1) edge (u3)
edge (u4);

\path (u2) edge (u3)
edge (u4);
\end{tikzpicture}
\qquad
\begin{tikzpicture}[roundnode/.style={circle, fill=black!60, very thick,minimum size=2mm, inner sep=1.2pt}]

%Nodes
\node[roundnode]      (u1)                     {};
\node[roundnode]      (u2)       [below=1cm of u1] {};
\node[roundnode]      (u3)       [below right=0.5cm and 0.8cm of u1] {};
\node[roundnode]      (u4)       [right=0.8cm of u3] {};
\node[]      (ellipses)       [right=0.15cm of u4] {\ldots};
\node[roundnode]      (u5)       [right= 1.4cm of u4] {};

%Lines

\path[dotted] (u1) edge (u2);

\path (u1) edge (u3)
edge (u4)
edge (u5);

\path (u2) edge (u3)
edge (u4)
edge (u5);
\end{tikzpicture}

	\caption{Subgraphs used in sufficient statistics of ERGMs.}
\end{figure}


\begin{definition}[Alternating $k$-star statistic \cite{Sni+06}]
	 \label{def:altkstar}
	The \emph{alternating $k$-star} statistic on graph $x$ with weighting parameter $\lambda \geq 1$ is defined as
	\begin{align*}
	u^{(s)}_\lambda(x) & = S_2 - \frac{S_3}{\lambda} + \frac{S_4}{\lambda^2} - \dots + (-1)^{n-2} \frac{S_{n-1}}{\lambda^{n-3}} \\
	&  = \sum_{k = 2}^{n-1} \frac{S_k}{\lambda^{k-2}}
	\end{align*}
\end{definition}

We introduce the notion of ``shared partners'' of two nodes -- the number of common neighbors that two nodes share -- which give a clean way to count $k$-triangles and $k$-two-paths.
\begin{definition}[Shared partners]
	\label{def:shared_partners}
	We denote the \emph{shared partner count} of nodes $i$ and $j$: 
	\begin{align}
	P_{ij}(x) = \sum_{\ell \in V} x_{i \ell} x_{j \ell}
	\end{align} 
\end{definition}

We define \emph{$k$-triangles} analogously to $k$-stars, so that a $k$-triangle consists of $k$ triangles that all share an edge. We can count the total number of $k$-triangles in a graph using the number of shared partners:
\begin{align}
\label{eq:k-triangle}
T_k(x) = \sum_{1 \leq i < j \leq n} x_{ij} \binom{P_{ij}}{k} \quad \text{for } (k \geq 2), \quad \text{and } T_1 = \frac{1}{3} \sum_{1 \leq i < j \leq n} x_{ij} P_{ij}
\end{align}
%where $T_1$ has an extra factor of $\frac{1}{3}$ in front because of the symmetry of a $1$-triangle for all three nodes included in the triangle.

\begin{definition}[Alternating $k$-triangle statistic \cite{Sni+06}]
	\label{def:altktri}
	The \emph{alternating $k$-triangle} statistic on graph $x$ with weighting parameter $\gamma \geq 1$ is defined as
	\begin{align*}
	u^{(t)}_\gamma(x) & = 3T_1 - \frac{T_2}{\gamma} + \frac{T_3}{\gamma^2} - \dots + (-1)^{n-3} \frac{T_{n-2}}{\gamma^{n-3}} \\
	&  = 3 T_1 + \sum_{k = 2}^{n-2} \left(\frac{-1}{\gamma}\right)^{k-1} T_k
	\end{align*}
\end{definition}

We define an \emph{independent $k$-two-path} as a pair of nodes (possibly connected or unconnected) with $k$ paths of length $2$ connecting them. We can think of a $k$-two-path as a precondition for a $k$-triangle, since every $k$-triangle must contain an independent $k$-two-path. In terms of shared partners, independent $k$-two-paths can be represented as:
\begin{equation}
\label{eq:k-two-path}
U_k(x) = \sum_{1 \leq i < j \leq n} \binom{P_{ij}}{k} \text{ for } k \not= 2 \quad \text{and } U_2(x) = \frac{1}{2} \sum_{1 \leq i < j \leq n} \binom{P_{ij}}{2}
\end{equation}
%where $U_2$ is preceded by a factor of $\frac{1}{2}$, because a $k$-two-path with $k=2$ is a $4$-cycle and hence is symmetric with respect to the two pairs of non-adjacent nodes making up the cycle.

\begin{definition}[Alternating $k$-two-path statistic \cite{Sni+06}]
	\label{def:altktwopath}
	The \emph{alternating $k$-two-path} statistic on graph $x$ with weighting parameter $\gamma \geq 1$ is defined as 
	\begin{align*}
	u^{(p)}_\gamma(x) & = U_1 - \frac{2 U_2}{\gamma}   + \frac{U_3}{\gamma^2} - \dots + (-1)^{n-3} \frac{U_{n-2}}{\gamma^{n-3}} \\
	&  = U_1 - \frac{2 U_2}{\gamma} + \sum_{k = 3}^{n-2} \left(\frac{-1}{\gamma}\right)^{k-1} U_k
	\end{align*}
\end{definition}

Now, having defined the ``alternating'' sufficient statistics, the proposed model has the form
\begin{equation}
\label{eq:ergm_alt_stats}
\Pr(x | \theta) = \exp\left\{\theta_1 E(x) + \theta_2 u_\lambda^{(s)}(x) + \theta_3 u_\gamma^{(t)}(x)  + \theta_4 u_\gamma^{(p)}(x) - \psi(\theta)  \right\}
\end{equation}

where $E(x)$ is the number of edges in graph $x$, the alternating k-two-path and k-triangle statistics generally use the same weighting parameter $\gamma$. In practice, a subset of the sufficient statistics can be used in the model, depending on what properties of a graph are pertinent to model for a given network. 

The overarching motivation behind introducing ``alternating'' sufficient statistics of the ERGMs is that these statistics are robust to addition or removal of an edge adjacent to an individual node, alleviating degeneracies  in the Markov graph model. For instance, consider adding an edge to a high degree node with degree $k$. This new edges contributes one $(k+1)$-star, $\binom{k}{k-1}$ $k$-stars, $\binom{k}{k-2}$ $(k-1)$-stars and so on. Therefore, the total number of additional stars in the graph resulting from adding this edge is $\sum_{i = 0}^{k} \binom{k}{i} = 2^k$. For Markov graphs including all stars with arbitrary associated parameters, this could lead to a large increase (or decrease) in the likelihood of the graph making the model degenerate as it places almost all of its probability on either near-complete or near-empty graphs. However, by imposing constraints on the parameters $\theta_k$, namely by alternating the signs of  the $k$-star statistics, the additional $(k-1)$-stars and $k$-stars balance each-other out.   The same general reasoning applies to the use of alternating statistics for $k$-triangles and $k$-two-paths -- alternation prevents the probability distribution from putting all of its mass on graphs with many high degree nodes,  preventing degeneracy of the model. %In particular, in the case of $k$-stars, we can think of the alternating $k$-star statistic as imposing constraints on the $\theta_k$ in \Cref{ex:markov_graphs}, namely that they must be alternating in sign and geometrically decreasing. In doing so, we enforce the property that adding edges to low degree nodes makes a significant difference in the likelihood of the graph, while adding edges to already high degree nodes makes less of a difference. 
 
 This interpretation of alternating statics as limiting the sensitivity of the likelihood to addition or removal of edges to high degree nodes can be understood by looking at an alternative representation of the statistics in terms of the degree distribution and the number of shared partners for nodes. Below, we present these equivalent representations of the statistics, which will also be helpful in proofs of privacy in \Cref{ch:dp_stats}. 
 
 \subsubsection{Alternating $k$-star}
 
 Note that using the relationship between $k$-stars and degrees given in \Cref{ex:markov_graphs} along with the binomial theorem we can rewrite the \emph{alternating $k$-star} statistic as:
 
 \begin{align}
 \label{eq:alternative-k-star}
 u_\lambda^{(s)}(x) & = \sum_{i = 1}^{n-1} D_i(x) \sum_{k = 2}^{n-1} \left(\frac{-1}{\lambda}\right)^{k-2} \binom{i}{k}   \nonumber\\
 & =  \lambda^2 \sum_{i = 0}^{n-1} \left(\frac{\lambda - 1}{\lambda}\right)^i D_i + 2 \lambda |E| - n \lambda^2
 \end{align}
The alternating $k$-star statistic is thus made up of the number of edges as well as a linear combination of the degree sequence where lower degree nodes are up-weighted exponentially compared to higher degree nodes, reflecting the tendency towards a power law degree distribution. Since a term representing the number of edges in the network is generally included along with this statistic, the model is mathematically equivalent to a model using a geometrically weighted average of the degree sequence. Sociologically, the coefficient of the $k$-star statistic can thus be interpreted as the propensity for high degree nodes in the network. If the coefficient of the statistic is positive, then networks with a few high degree ``hubs'' are observed, while if it is negative, high degree nodes are discouraged and the network consists of mostly low-degree nodes \cite{Sni+06}.
 
\subsubsection{Alternating $k$-triangle}
 
 Similarly, for the alternating $k$-triangle statistic, we can gain insight by rewriting in terms of the number of shared partners for pairs of nodes. By using this representation of $k$-triangles from \Cref{eq:k-triangle} along with the binomial theorem, we can rewrite the \emph{alternating $k$-triangle} statistic as:
\begin{align}
 \label{eq:alternative-k-tri}
u_\gamma^{(t)}(x) & = \sum_{1 \leq i < j \leq n} x_{ij} \sum_{k = 1}^{n-2} \left(\frac{-1}{\gamma}\right)^{k-1}  \binom{P_{ij}}{k} \nonumber\\
& = \gamma \sum_{1 \leq i < j \leq n} x_{ij} \left(1 - \left(\frac{\gamma-1}{\gamma} \right)^{P_{ij}} \right) \nonumber\\
& = \gamma |E| - \gamma \sum_{1 \leq i < j \leq n} x_{ij} \left(\frac{\gamma-1}{\gamma} \right)^{P_{ij}}
\end{align}
Note, then, that any edge that does not participate in a triangle (so $x_{ij} = 1$ but $P_{ij} = 0$) does not contribute to the the alternating $k$-triangle statistic. On the other hand, as we add additional shared partners to an edge, the second term in (\ref{eq:alternative-k-tri}) falls exponentially so the statistic increases, but  by less for higher order $k$-triangles than for lower-order triangles.  Sociologically, this term can be interpreted as the importance of triangle closure in the generation of the graph \cite{GKM09}. In contrast to directly including the number of triangles in the graph, the alternating $k$-triangles statistic is more stable, preventing the model degeneracies discussed above. 

\subsubsection{Alternating $k$-two-path}

Using the representation of $k$-two-paths in terms of shared partners from \Cref{eq:k-two-path} and the binomial theorem, we can rewrite the \emph{alternating $k$-two-path} statistic as: 
\begin{align}
\label{eq:alternative-k-two-path}
u_\gamma^{(p)}(x) & = \sum_{1 \leq i < j \leq n} \sum_{k = 1}^{n-2} \left(\frac{-1}{\gamma}\right)^{k-1}  \binom{P_{ij}}{k} \nonumber\\
& = \gamma \sum_{1 \leq i < j \leq n}\left(1 - \left(\frac{\gamma-1}{\gamma} \right)^{P_{ij}} \right) \nonumber\\
& = \gamma \binom{n}{2} - \gamma \sum_{1 \leq i < j \leq n} \left(\frac{\gamma-1}{\gamma} \right)^{P_{ij}}
\end{align}
Thus, the alternating $k$-two-path has an interpretation similar to that of the alternating $k$-triangle. As shared partners are added for any two nodes, the second term of the statistic increases, but the increase falls exponentially with additional partners. This term is generally only included in conjunction with the $k$-triangle statistic to try to separate out the effects of two-paths forming between unconnected nodes and mutual connections forming between already connected nodes.

\subsection{Sufficient Statistics for Labeled Nodes}

The alternating statistics over $k$-stars, $k$-triangles and $k$-two-paths capture structural properties of network data. Frequently, however, there are labels associated with nodes in the network, which are important to model. ERGMs can take into consideration both the structure of the network and the labels associated with nodes, by including sufficient statistics based off of the labels, allowing researchers to capture properties like homophily, the tendency for similar actors to build relationships with one another within a network. Generally, labels are taken to be fixed and exogenous to the edges, so that attributes of the nodes may affect the formation of the network, whereas relationships in the network are not thought of as impacting attributes. This is generally a reasonable assumption, as labels often represent the identity of an individual, containing characteristics like gender, race, or age. As there are many potential ways to incorporate labeled data into an ERGM, we will focus here on three of the most commonly used statistics for discrete nodal attributes, ``homophily'', ``popularity'' and ``mixing.''\footnote{These have fairly straightforward analogues for continuous nodal attributes, but we focus on the discrete case, as this is applicable to the dataset analyzed.} 

In particular, letting $z_i$ be a discrete attribute of node $i$ (gender, for instance) we introduce the following sufficient statistics to represent different processes of social selection \cite{LKR12}:

\begin{table}[H]
	\caption{Common sufficient statistics for discrete nodal attributes.}
	\label{table:node_suff_stats_defn}
	\begin{center}
	\begin{tabular}[!h]{ll}
	\hhline{=|=}
	\multicolumn{1}{c}{\emph{Parameter}} &  \multicolumn{1}{c}{\emph{Statistic}}  \\ \midrule
	Homophily (Uniform)  & $\sum\limits_{i < j} x_{ij} \I (z_i = z_j) $ \\\midrule
	Homophily (Differential)  & $\sum\limits_{i < j} x_{ij} \I (z_i = z_j = a) $ \\\midrule
	Popularity &  $\sum\limits_{i < j} x_{ij} \left(\I (z_i = a) + \I (z_j = a) \right) $ \\ \midrule
	Mixing & $\sum\limits_{i < j} x_{ij} \I (z_i = a) \I (z_j = b)$  \\
	\hhline{=|=}
	\end{tabular}
	\end{center}
\end{table}

Uniform homophily captures the tendency for nodes with the same attribute to share an edge, while differential homophily captures this phenomenon for a specific attribute, which may be useful if, for instance, we thought that men and women have different propensities to become friends with people of the same gender. The popularity parameter is fairly self-explanatory as it measures the number of edges that have nodes with a given attribute as an endpoint and can be thought of as the overall sociability of a group with a specific attribute. Finally, the mixing parameter represents the number of edges between nodes with two different, specific attributes. Including such nodal attribute statistics in conjunction with the alternating sufficient statistics discussed in \Cref{sec:alt_stats} allows for specification of ERGMs that separate out social selection effects like homophily from structural effects like triangle closure, making ERGMs a powerful modeling tool.

\section{Bayesian Inference on ERGMs}

Having provided an ERGM specification that captures the characteristics of interest in a network, the goal of inference is to find parameters $\theta$ that describe the realized data well. In the framework of maximum likelihood estimation, this means finding a $\theta$ that maximizes the probability of drawing observed network $x_{obs}$ from the distribution $p(X|\theta)$. In the Bayesian paradigm, an analyst specifies a prior distribution over $\theta$ and then wishes to compute a posterior distribution of $\theta$ given the observed network. Bayesian inference is more general than maximum likelihood inference in the sense that if an analyst chooses a flat prior on $\theta$ (a uniform prior over the parameter space) and takes the maximum of the posterior as a point estimate, then Bayesian inference reduces to maximum likelihood inference.
%By the theory of exponential family distributions, this $\hat{\theta}_{MLE}$ corresponds to the probability distribution that centers $x_{obs}$ (that is, $\hat{\theta}_{MLE}$ is the parameter $\theta$ such that $\E_{\theta}[u(X)] = u(x_{obs})$.) 

In general, exact inference is not feasible for ERGMs due to the presence of the intractable normalizing constant $\psi(\theta)$ in the likelihood (\Cref{def:ergm_defn}), which is a sum over the space of possible graphs on $n$ nodes of size $2^{\binom{n}{2}}$. Therefore, a number of approximate MCMC approaches have been proposed to perform inference. In this work, we focus on Bayesian inference over ERGMs, because it constitutes the state-of-the-art in non-private inference methods and has been shown to be more stable than MCMC-MLE approaches \cite{CF11}. Additionally,  the noise from differentially private mechanisms can be incorporated quite naturally into the Bayesian framework.  The non-private Bayesian inference method proposed by Caimo and Friel is based on the Exchange Algorithm\cite{MGM12} and is fairly simple to describe: 

\begin{algorithm}
	\caption{Non-Private Bayesian Inference for ERGMs (Exchange Algorithm) \cite{CF11} }
	Input: ERGM distribution $\pi(X | \theta)$, prior $p(\theta)$, observed graph $x_{obs}$, number of burn-in draws $r$, symmetric proposal distribution $h(\cdot | \theta)$. \\
	Output: sequence of draws $\theta^{(r)},...\theta^{(T)}$ from posterior distribution $p(\theta | x_{obs})$.
	
	\vspace{0.1in}
	For {$ t = 1,...,T$}:
	\begin{enumerate}
		\item  Draw parameter vector $\theta^* \sim h(\cdot | \theta^{(t-1)})$
		\item Sample graph $x^* \sim \pi(\cdot | \theta^*)$
		\item Accept the proposed move with probability $\min\left\{1, \alpha\right\}$. If the move is accepted, set $\theta^{(t)} = \theta^*$. Otherwise, set $\theta^{(t)} = \theta^{(t-1)}$
	\end{enumerate}
	where \begin{align*}
	\alpha &= \frac{p(\theta^*)}{p(\theta^{(t-1)})} \exp\left\{\left(\theta^* - \theta^{(t-1)}\right)^T\left(u(x_{obs}) - u(x^*)\right) \right\}
	\end{align*}
\end{algorithm}

The algorithm can be justified by considering sampling from an augmented distribution with two auxiliary variables $x^*, \theta^*$:
$$p(x^*, \theta^*, \theta | x_{obs}) \propto \pi(x_{obs} | \theta)p(\theta) h(\theta^* | \theta) \pi(x^* | \theta^*) $$
where $\pi$ refers to the ERGM probability distribution. Marginalizing out $\theta^*$ and $x$ from the augmented distribution gives the posterior distribution $p(\theta | x_{obs})$ of interest. Steps $1$ and $2$ are Gibbs updates of $\theta^*$ and $x^*$, while step $3$ can be justified as the appropriate Metropolis-Hastings acceptance ratio:
\begin{align*}
\alpha & = \frac{ \pi(x_{obs} | \theta^*)p(\theta^*) h(\theta^{(t-1)}| \theta^*) \pi(x^* | \theta^{(t-1)})  } { \pi(x_{obs} | \theta^{(t-1)})p(\theta) h(\theta^* | \theta{^{(t-1)}})   \pi(x^* | \theta^*)  } \\
& = \frac{p(\theta^*)}{p(\theta)} \frac{\pi(x_{obs} | \theta^*) \pi(x^* | \theta^{(t-1)})}{ \pi(x_{obs} | \theta^{(t-1)}) \pi(x^* | \theta^*) }
\end{align*}
where we drop the $h$ transition probabilities by symmetry and the intractable normalizing constants for $\theta^*$ and $\theta^{(t-1)}$ cancel, allowing easy computation of $\alpha$. Thus, by standard MCMC theory, the draws $\theta^{(t)}$ come asymptotically from the desired posterior distribution.

In practice, Caimo and Friel advocate the use of a population-MCMC variant of their basic algorithm, in which multiple Markov chains are run in parallel, with the state space defined over the $\theta$'s of these multiple chains, as this population MCMC approach tends to converge faster and lead to less temporal dependence in draws from the Markov chain.  We use this method, known as Parallel Adaptive Direction Sampling, in our private inference methods and explain it in detail in \Cref{appendix_mcmc} along with a Metropolis-Hastings sampler to simulate networks from an ERGM with specified parameters.




